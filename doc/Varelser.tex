\section{Varelser}
%
\subsection{Leopardarm}
\label{var:Leopardarm}
%
\begin{center}
	\rowcolors{2}{gray!25}{white}
	\begin{tabular}{ | c | c | }
		\hline
		Egenskap & Värde \\
		\hline
		STY & 95 \\
		FYS & 50 \\
		STO & 10 \\
		SMI & 95 \\
		VST & 65 \\
		\hline
	\end{tabular}
\end{center}
%
\textbf{KP:} 4 \quad \textbf{Skadebonus:} +1T4 \\
\textbf{Kroppsbyggnad:} 0 \quad \textbf{Förflyttning:} 0 \\
\\
En \textit{leopardarm} är en åkallad åkallelse från Prof. Olofsson som materialiseras från en tegelvägg och tar tag i en utredare. Den kommer att börja riva och bita i utredarens hals och försöker sedan växa ihop med dennes upprivna kött. Det sker i den här ordningen, en per spelrunda. Under varje spelrunda som utredaren är vid medvetande kan denne försöka ta sig fri från greppet. Det blir ett styrkeslag mellan utredaren och armen. Om armen blir skadad kommer den börja röra sig och skaka utredaren, men den kommer inte att försvara sig.

\begin{enumerate}
	\item Klor och tänder börjar röra sig över armen i en sågrörelse, och utredaren börjar känna en brännande känsla av att deras kött börjar växa ihop. Utredaren tar 1 KP skada. Armen kan fortfarande dras av utan att det orsakar mycket smärta för utredaren, men denne kommer fortfarande ha en uppskuren hals.
	\item Köttet börjar sitta ihop. Blodådror börjar formas mellan de olika kropparna. Utredaren tar 1T2 KP i skada. Armen kan fortfarande slitas av med lite kraft, men det kommer göra mycket ont. Isåfall måste utredaren lyckas med ett normalt slag för viljestyrka för att inte svimma.
	\item Nerver börjar formas mellan de olika kropparna och utredaren börjar känna den oerhörda smärtan från blottat kött emot tegel, samt all fysisk skada som görs på armen. Utredaren tar 1T4 skada. Armen måste nu avlägsnas kirurgiskt, eller slitas av med ett lyckat svårt slag för styrka. Då måste utredaren lyckas med ett svårt slag av viljestyrka för att inte svimma. Utredaren tar också 1T4 i skada.
	\item Armen är nu en del av utredaren. Den måste nu avlägsnas kirurgiskt, men om någon lyckas med ett extremt svårt slag för styrka kan de slita av den. Det skulle kännas som att slita ut någons adamsäpple, och utredaren måste klara ett extremt svårt slag för viljestyrka för att inte svimma. Den tar också 2T4 i skada.
	\item Armen växer in i utredarens luftrör, och denne börjar kvävas. Utredaren kommer att kvävas efter fem spelrundor. Denne får slå ett slag för viljestyrka för att inte svimma, och måste klara stegvis ökande svårighetsgrader för varje runda. Efter fem spelrundor är utredaren död.
\end{enumerate}

När armen går till 0 KP kommer den bli livlös. Om den suttit på utredaren mer än två spelrundor kommer den fortfarande vara vid liv, men handlingslös tills den avlägsnas från halsen. Annars kommer den börja likvideras till en klump av kött och ben utsmetat på väggen och marken.


\subsection{Leopard (Homunculus)}
\label{var:Leopard}
%
\begin{center}
	\rowcolors{2}{gray!25}{white}
	\begin{tabular}{ | c | c | }
		\hline
		Egenskap & Värde \\
		\hline
		STY & 95 \\
		FYS & 50 \\
		STO & 10 \\
		SMI & 95 \\
		VST & 65 \\
		\hline
	\end{tabular}
\end{center}
%
\textbf{KP:} 13 \quad \textbf{Skadebonus:} 3 \\
\textbf{Kroppsbyggnad:} 2 \quad \textbf{Förflyttning:} 10 \\
\\
\textbf{Attacker per runda:} 2 \\
\textbf{Stridsattacker}
\begin{itemize}
	\item Strid 60\%, skada 2T6 + 3
	\item Ducka 25\%
\end{itemize}
\textbf{Rustning:} 2 poäng päls och hud. \\
\textbf{Färdigheter:}
\begin{itemize}
	\item Smyga 30\%
	\item Spåra 35\%
\end{itemize}
%
En \textit{leopard homonculus} är en kropp, lik en leopard, som har animerats av Prof. Olofsson. Dessa agerar mycket som vanliga leoparder, men de klarar mycket mer stryk och kan fortsätta strida även om de är lemlästade. När en leopard homonculus dör smälter de ihop till en sörja av blod och ben. Professorn kan bara åkalla en leopard per dag, igenom en utförlig ritual.
%
\subsection{Buse}
\label{var:Buse}
%
\begin{center}
	\rowcolors{2}{gray!25}{white}
	\begin{tabular}{ | c | c | }
		\hline
		Egenskap & Värde \\
		\hline
		STY & 65 \\
		FYS & 50 \\
		STO & 45 \\
		SMI & 40 \\
		INT & 30 \\
		VST & 35 \\
		\hline
	\end{tabular}
\end{center}
%
\textbf{KP:} 9 \quad \textbf{Skadebonus:} 0 \\
\textbf{Kroppsbyggnad:} 0 \quad \textbf{Förflyttning:} 8 \\
\\
\textbf{Färdigheter}:
\begin{itemize}
	\item Hota 50\%
	\item Strid (Handgemäng) 35\%
	\item Smyga 40\%
	\item Kasta 30\%
	\item Språk (Svenska) 16\%
\end{itemize}
