\section{Övriga platser}
\subsection{Servicedesk}
\label{loc:Servicedesk}
I hus B på Luleå Universitet kan man gå till \textit{Servicedesk}, något utav en lobby för universitetet. Där jobbar en dam vid namn \textit{Karolin Markesjö}, en äldre kvinna som fungerar som hjälpperson för de som undrar saker om universitetet. Hon kan hjälpa äventyrarna att hitta professorer inom de fält som de undrar om. Här är en lista på några. Ett - i deras område syftar på att de fungerar som en placeholder för spelledaren att snabbt skapa en professor inom något abstrakt fält.

\begin{center}
	\label{tab:Professorer}
	\begin{tabular}{ | l | l | l |  }
		\hline
		\multicolumn{3}{|c|}{Professorer} \\
		\hline
		Namn & Fält & Rum \\ 
		\hline
		Stefan II Eriksson & Mattematik & E890 \\
		Valdemar Ros & Fysik & E138 \\
		Anders Ros & Appl. Mekanik & E142 \\
		Björn Ekdahl & Zoologi & A142 \\
		Göran Wall & Ornitologi & A320 \\
		Simon Abreo & Botanik & A342 \\
		Henrik Carlsson & Ekologi & A109 \\
		Elbert Seidel & Europeisk och & \\ 
		 & Nordamerikansk Historia & F295 \\
		Urban Tornvall & Svensk Urhistoria & F213 \\
		Benjamin Klip & Svensk och Nordeuropeisk kultur & F239 \\
		Holger Petersson & Skandinavisk religion & F152 \\
		Migele Chapdelain & Litterär historia & F114 \\
		Tomas Olofsson & Asiatisk och Afrikansk religion & F103 \\
		Rune Sahl & Konsthistoria & F301 \\
		Denis Kleyko & Kommunikationsteori & F194 \\
		Bertil Säf & Modern Filosofi & F203 \\
		Christer Stocker & - & - \\
		Roger Birk & - & - \\
		Gustav Unrot & - & - \\
		\hline
	\end{tabular}
\end{center}

Här är också lite övrig personal.

\begin{center}
	\label{tab:OvrigPersonal}
	\begin{tabular}{ |p{3cm}|p{3cm}|p{3cm}|  }
		\hline
		\multicolumn{3}{|c|}{Övrig personal} \\
		\hline
		Namn & Fält & Rum \\ 
		\hline
		Börje Verhn & Rektor & B382 \\
		Karl Hansson & Vaktmästare & B103 \\
		Ingrid Hansson & Städerska & D121 \\
		Ebba Björnberg & Städerska & D121 \\
		\hline
	\end{tabular}
\end{center}


\subsection{Professorers kontor}
%
Här är några kontor du kan använda om Utredarna går till någon av professorerna, förutom Prof. Migele Chapdelain eller Prof. Tomas Olofsson.

\subsubsection*{Kontor 1}

Utredarna ser ett mörkt rum, fyllt med kartonger och diverse bråte. De mörkt röda persiennerna är nerdragna, och när ni känner på dörren är den låst. Om Utredarna väljer att dyrka upp dörren är det ett svårt slag, då universitetet använder säkra lås för sina kontor. I kontoret kan dock utredarna finna dokument och mindre verktyg som korresponderar med professorns fält, samt en en mindre kniv använd för att öppna kartonger samt brev.

\subsubsection*{Kontor 2}

Utredarna kommer till ett kontor med ett skrivbord centrerat i rummet. Bakom skrivbordet sitter professorn och läser i en tidskrift. Om någon undersöker tidskriften kan man lätt se att det är den senaste volymen av \textit{Luleå studentkårs magasin}. På skrivbordet står en skrivmaskin och en massa papper samt böcker är utspridda runt omkring.

\subsubsection*{Kontor 3}
Utredarna ser ett rum med skrivbord, hylla samt en soffa som alla är intryckta emot väggarna. I mitten av rummet står någonting relaterat till professorns fält. Det kan vara:
%
\begin{itemize}
	\item En stor maskin som är avslagen.
	\item Ett terrarium för djur/växter
	\item En ofärdig staty
	\item En elektrisk maskin, med massa reläer och lampor (En gammaldags dator)
\end{itemize}
%
I rummet håller professorn på med sitt projekt. Om utredarna frågar professorerna vet de detta:
%
\begin{itemize}
	\item De vet att Anna har dött, och beklagar utredarna över detta. Hennes farbror Prof. Olofsson är förkrossad över detta och har tagit dagen ledig.
	\item Prof. Chapdelain är en oklar man. De brukar inte se honom runt sitt kontor, förutom här om dagen när hans fönster hade funnits krossat.
	\item Annas sår vet de inget om, men de föreslår att de ska prata med Prof. Carlsson som är ekolog, eller Prof. Olofsson själv som har spenderat en del tid med att jaga lejon i Afrika.
\end{itemize}
