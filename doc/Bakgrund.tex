\section{Bakgrund}
Luleå Universitet är (i det här fiktiva universumet) ett universitet som anlades utanför Luleå år 1886. De institutioner som är främst aktiva här är Institutionen för naturhistoria, där botanik och zoologi forskas om, Institutionen för teknik, som sysslar med mattematik, mekanik och fysik, och Institutionen för filosofi, där allt mellan historia, religion, samhällskunskap och konsthistoria studeras. Det mycket få vet om är att det även finns en till institution, Institutionen för signeri- och rationellt avvikande teknik, som sysslar med att dokumentera det ockulta, studerar \textit{taumaturgi} (Läran om magi, ritualer och transmutation) samt samlar på övernaturliga artefakter och verk.

Den 16 december 1912 hölls en rättegång i byn Bangbama, Sierra Leone i sydvästra Afrika. Rättegången var emot en stor skara människor från olika byar runtomkring, som alla var medlemmar i det fruktade \textit{Leopardsällskapet}. Dessa individer tillbad ett föremål som de kallade \textit{Borfima}, ett blodsoffer från innehavarens egna familj.

Äventyret utspelar sig på Luleå Universitet våren 1913, och utredarna i denna kampanj är alla studenter vid Luleå Universitet. De har lärt känna varandra igenom \textit{Anna Olofsson} \sectiondescribe{\ref{kar:AnnaOlofsson}}, en litteraturstudent som också är ansvarig utgivare för \textit{Luleå studentkårs magasin}. Det hela börjar med att de blir inbjudna till Annas lägenhet på middag. Natten efter detta upplever alla äventyrare samma dröm, och när de väcks av en polis vid deras dörr får de reda på att Anna har funnits död, med stora klösmärken över hennes strupe...

\subsection{En kommentar om äventyret}
Detta äventyr, som de flesta andra Call of Chtulu-äventyr, utspelar sig under en mycket problematisk tid. Universitetets anställda professorer är alla män, vilket jag låtit hållas enligt denna tidsålder. Jag ber om ursäkt om någon finner detta stötande.

Äventyret infattar även kommentarer angående människor från Afrika. Även fast de mer anstötliga tidsenliga namn för dessa skulle kunna användas, har jag ändå valt att referera till dem i det jag anser som den minst problematiska versionen för de tidsenliga uttrycken. De nämns i det här äventyret som \textit{svarta} eller \textit{den svarte mannen}. Jag ber om ursäkt om även detta anses anstötligt, men det är den kompromiss som jag valt.

\section{Äventyrets början}

\begin{displayquote}
	\textit{Luleå Universitet} är Norrlands metropol när det kommer till den akademiska världen. Trots att antalet bildade i den norra delen av Sverige är mycket låg, beslöt sig självaste \textit{Gustaf VI Adolf} att ett universitet ska byggas i Sveriges nordligare del. Detta för att underlätta både de biologiska studier som utförs på Norrlands flora och fauna, samt för att bidra med teknisk kompetens till alla närliggande industrier.

	Ni är några av de få som valt att flytta ända upp till denna avlägsna plats, för att studera vid detta universitet. Men för en kväll har ni alla valt att skjuta studierna åt sidan för att medverka på den middag som er vän bjudit er till. \textit{Anna Olofsson} är en ung dam som studerar litteratur vid Luleå Universitet och är ansvarig utgivare för \textit{Luleå studentkårs magasin}, en tidskrift ämnad åt studerande och anställda vid Luleå Universitet. Ni träffar alla varandra vid trappuppgången till hennes studentlägenhet.
\end{displayquote}

\subsection{Annas Middag}

Äventyrarna blir snart insläppta i Annas lägenhet, en tvåa med en kokvrå, ett vardagsrum och ett sovrum. De serveras en stek med potatis och kokta grönsaker och Anna börjar småprata. Hon nämner att hon haft sådana problem på tidningens redaktion, de har fått en skokartong med en död råtta i levererat till deras entré. Hon misstänker att de är ``De satans ungarna från Porsöligan igen''. De har uppenbarligen trakasserat tidskriftens arbetare flera gånger efter att de publicerat ett avslöjande som fick en av medlemmarna att hamna i fängelse. När äventyrarna lämnar Anna verkar hon sömnig, men ändå väldigt glad över sin middag. 

\subsection{Utanför Annas lägenhet}
När ni kommer ut från lägenheten är det mörkt ute. Det ni ser är upplyst av de gatulyktor som finns placerade på den gångväg som är upplyst runt det här kvarteret av lägenheter.

Alla utredare får slå ett slag för Finna dolda ting. Om denne lyckas med ett svårt slag kan de se en skugga, som snabbt hukar sig bakom en husvägg, ca 30 meter bort. Om utredarna springer mot skuggan, märker de snabbt att den individ som spionerat på dem är borta.

\subsection{Mordet}
Sedan när alla sover.

\begin{displayquote}
	Under natten kommer det till er en dröm. Ni befinner er alla, själv, i ett mörkt rum fullt av dimma. Ni ser ett stenlagt golv, som ni verkar stå hukande på, som att ni är helt utmattade. Ni kan inte urskilja hur de ser ut, men ni ser två skuggiga människor som ser ner på er. Den ena säger ``Sök vidare. Vad vet hon egentligen?''.

	Ni vaknar kallsvettiga av att någon knackar på ytterdörren.
\end{displayquote}
%
Utanför era dörrar står en poliskonstapel. Han hämtar utredarna en åt gången, och ber dem följa med honom till stationen. Börja med en utredare, där konstapeln plockar upp denne och sedan går till nästa utredare (med den första i följe), och fortsätt till alla utredare. Konstapeln introducerar sig som \textit{Carl Svidsstål} \sectiondescribe{\ref{kar:KonstapelCarlSvidstal}} och är mycket tydlig med att utredarna inte är anhållna, utan att detta endast är ett förhör. 

Utredarna leds alla till en bil (eller två) som för er till Luleås Polishus. Där förs de in i ett väl upplyst rum med ett bord och stolar. Konstapel Svidstål kommer snart in med en portfölj. Han ser allvarlig ut, men också något sorgsen ut.

Han undrar vart utredarna var igår, när de senast såg Anna Olofsson, och vad alla gjorde när det lämnat hennes lägenhet. Konstapeln berättar med en sorgsen röst att det skett ett mord, och att offret är utredarnas vän Anna. Han tar upp sin portfölj, och berättar för utredarna att de absolut inte måste se på de bilder han har, men att han förstår att det kan vara viktigt för att förstå allvaret i situationen.

\begin{displayquote}
	På den första bilden ser ni Annas kropp, livlös på en brits. Hennes vanligen så vackra ansikte är blekt och insjunket. Bilden infattar hennes huvud ner till hennes bröstkorg, och ni kan se de fasansfulla snitt som går tvärs över hennes hals. Det är tre djupa sår som går parallellt med varandra, med ett mellanrum på tre centimeter mellan varje jack.

	Den andra bilden föreställer Annas ben, visat bakifrån. Man kan se ännu en skada, men här verkar det vara ett mycket större ingrepp. Baksidan av hennes lår är borta, och det som finns kvar är ett stort köttsår från nedre delen av hennes stjärt ner till hennes knäveck. Ni kan se att det blod som än gång pumpats igenom hennes värmande hjärta, nu har koagulerat och torkats bort. Det enda som syns är hennes kött och vitan av hennes lårben.
\end{displayquote}
%
Alla utredare som ser på bilden tappar 1/1T4 + 1 i sinneshälsa. 

Konstapel Svidsstål beklagar ypperligt, och säger till utredarna att de är fria att gå. Han uppmanar dock till att utredarna inte ska gå till brottsplatsen, då Anna inte längre förvaras där. Ni skjutsas åter hem till era lägenheter. Har utredarna svårt att besluta vad de ska göra kan du låta dem slå ett idéslag. Lyckas de med ett normalt slag kan du tipsa dem om att Anna nämnt att hennes tidskrift haft problem med Porsöligan. De kan antingen försöka lista ut vart Porsöligan håller hus, eller gå till redaktionen. Lyckas de med ett svårt slag kan du även tipsa dem om att det kan finnas professorer som vet något om den sortens skador som Anna har fått, och att det bästa sättet att få kontaktuppgifter till en professor är att gå till \textit{Servicedesk} i Hus B \sectiondescribe{\ref{loc:Servicedesk}}.