\section{Brådmans Bar}
\label{loc:BradmansBar}
%
\begin{displayquote}
	Luften i baren är fylld av dova samtal och röken från cigaretter. Bakom den stora ek-baren står en herre och verkar vänta på kunder. Han stirrar på er när ni träder in. Diverse folk sitter runt olika runda bord. Många ser ut som studenter, men det är även en hel del övrigt folk här.
\end{displayquote}
%
Om utredarna lyckas med ett svårt slag för psykologi kan de märka att det verkar vara en dålig stämning vid ett bord, mellan en grupp ungdomar och en äldre herre. De verkar väsa kalt gentemot varandra och argumenterar intensivt. Låt utredarna göra vad de vill ett tag i baren. Efter en stund händer följande.

\begin{displayquote}
	Plötsligt reser sig en äldre herre vid ett av borden hastigt. Han verkar börja röra sig mot utgången, men de tre yngre männen vid hans bord reser sig och hindrar honom. En stark röst från en av dem säger ``Vart fan tror du att du är påväg?''. Den äldre herren säger, med kraftig fransk brytning ``Om ni inte kan fullfölja er del av avtalet så är det här samarbetet över!''. Mannen försöker återigen röra sig mot entrén, men de yngre männen blockerar fortfarande vägen för dem.
\end{displayquote}
%
Låt utredarna slå ett slag för att finna dolda ting, och om de lyckas med ett normalt slag kan de se att en lapp faller ut ur fickan på mannen när han knuffas till.
Väljer utredarna att ingripa drar en av männen en kniv och säger ``Det här rör inte er!''. Under en strid kommer männen att dra sig ifrån baren om de känner sig övermannade, eller om utredarna lyckas hota dem. 

Mannen är Professor Migel Chapdelain \sectiondescribe{\ref{kar:MigelChapdelain}}. Hjälper utredarna professorn är han tacksam och bjuder dem alla på varsin drink. Frågar utredarna honom om vad han och de yngre männen håller på med säger han att ``De där busarna har varit och vandaliserat fönstret till mitt kontor. Jag ville sätta dem på sin plats''. Han sitter gärna och samtala med utredarna en stund om de vill. Han påstår dock inte att han vet någonting om vare sig mordet eller något mystiskt runt universitetet. Han är bra på att ljuga, vilket betyder att utredarna behöver lyckas med ett svårt slag för psykologi för att förstå att han inte berättar hela sanningen.

Lyckas utredarna fånga och hota en av de yngre männen kan de berätta att Prof. Chapdelain anlitat dem för att spionera på Anna Olofsson. Om utredarna fokuserar på busarna lyckas dock professorn smita iväg.
%
\subsection{Utanför Baren}
\label{loc:UtanfarBaren}

\begin{displayquote}
	Dörren ut ur baren leder till en mörk gränd mellan huset där baren ligger och Porsöns handelsbod. Det enda ljuset i gränden är det värmande skenet från glaset i barens entré. Den murade gången leder både till en öppning söderut, mot en bilparkering, och norrut mot handelsboden.
\end{displayquote}
%
Om utredarna låtit Prof. Chapdelain rusa ut från baren står han vid dörröppningen och röker. Han verkar något uppjagad, men vill utredarna prata med honom kan du ge dem dialogen från föregående paragraf, men han är något mer tillbakadragen och mindre villig att berätta saker. Sedan lämnar han utredarna. 

När han har gått kan en av utredarna antingen luta sig mot en vägg. Om de vägrar, låt denne bli tagen av monstret från kullerstenen. Låt utredarna (Utom den valda) slå för att finna dolda ting och om de lyckas med ett svårt slag (Mörk gränd, svårt att se) ser de det följande, mycket fort. Annars ser det detta samtidigt som nästa beskrivning.

\begin{displayquote}
	Du ser, i skårorna av kullerstenen bakom den valda utredaren, att något börjar röra sig. Det rinner någonting rött trögflytande ner för väggen. Vätskan klumpar ihop sig på olika ställen. Klumparna verkar forma något. Plötsligt spricker en tunn hinna, och innehållet i en av dessa bölder träder fram. Det är en klump av hår insmetat i blod och var som rör sig ner för väggen. Trådar formas i vätskan, och flätas ihop till fibrer av muskler. Bitar av ben träder fram och tar formen av en läm.
\end{displayquote}
%
Om utredarna ser detta och varnar den mot dörren får den möjlighet att slå ett slag för att ducka och kan lyckas med att ta sig bort från väggen med ett lyckat svårt slag. Annars händer följande.

\begin{displayquote}
	Du känner att en varm fläck formas på baksidan av din axel. Kanske har du lutat dig i något som satt fast på väggen? Precis innan du hinner vända dig om för att titta bakom dig kommer en arm fram, lägger ett grep omkring din hals och håller dig mot väggen. Det du hinner uppfatta är att den är blodig och slemmig, nästan som att du skulle flå någons arm medan denne fortfarande lever. Du ser fläckar av päls som verkar växa fläckvis på armen. Flikar av hud börjar ta form, och vecklar ut sig över din hals. Du känner en brännande smärta då klor och tänder tar form längst med armen, och färdas som en såg över din hals. Dess blod blandas med ditt, och du känner hur huden från armen lägger sig över dina sår och växer ihop med ditt kött.
\end{displayquote}

Armen som håller i utredaren är en \textit{Leopardarm} \sectiondescribe{\ref{var:Leopardarm}}, och kommer att hålla fast utredaren tills den förlorat alla sina kroppspoäng. Den kommer också att riva och bita i utredaren, och försöka växa samman med dennes hals.