\section{Servicedesk}
\label{loc:Servicedesk}
I hus B på Luleå Universitet kan man gå till \textit{Servicedesk}, något utav en lobby för universitetet. Där jobbar en dam vid namn \textit{Karolin Markesjö}, en äldre kvinna som fungerar som hjälpperson för de som undrar saker om universitetet. Hon kan hjälpa äventyrarna att hitta professorer inom de fält som de undrar om. Här är en lista på några. Ett - i deras område syftar på att de fungerar som en placeholder för spelledaren att snabbt skapa en professor inom något abstrakt fält.

\begin{center}
	\label{tab:Professorer}
	\begin{tabular}{ | l | l | l |  }
		\hline
		\multicolumn{3}{|c|}{Professorer} \\
		\hline
		Namn & Fält & Rum \\ 
		\hline
		Stefan II Eriksson & Mattematik & E890 \\
		Valdemar Ros & Fysik & E138 \\
		Anders Ros & Appl. Mekanik & E142 \\
		Björn Ekdahl & Zoologi & A142 \\
		Göran Wall & Ornitologi & A320 \\
		Simon Abreo & Botanik & A342 \\
		Henrik Carlsson & Ekologi & A109 \\
		Elbert Seidel & Europeisk och & \\ 
		 & Nordamerikansk Historia & F295 \\
		Urban Tornvall & Svensk Urhistoria & F213 \\
		Benjamin Klip & Svensk och Nordeuropeisk kultur & F239 \\
		Holger Petersson & Skandinavisk religion & F152 \\
		Migele Chapdelain & Europeisk religion & F114 \\
		Tomas Olofsson & Asiatisk och Afrikansk religion & F103 \\
		Rune Sahl & Konsthistoria & F301 \\
		Denis Kleyko & Kommunikationsteori & F194 \\
		Bertil Säf & Almänn filosofi & F203 \\
		Christer Stocker & - & - \\
		Roger Birk & - & - \\
		Gustav Unrot & - & - \\
		\hline
	\end{tabular}
\end{center}

Här är också lite övrig personal.

\begin{center}
	\label{tab:OvrigPersonal}
	\begin{tabular}{ |p{3cm}|p{3cm}|p{3cm}|  }
		\hline
		\multicolumn{3}{|c|}{Övrig personal} \\
		\hline
		Namn & Fält & Rum \\ 
		\hline
		Börje Verhn & Rektor & B382 \\
		Karl Hansson & Vaktmästare & B103 \\
		Ingrid Hansson & Städerska & D121 \\
		Ebba Björnberg & Städerska & D121 \\
		\hline
	\end{tabular}
\end{center}


\section{Professorers kontor}
%
Här är några kontor du kan använda om Utredarna går till någon av professorerna, förutom Prof. Migele Chapdelain eller Prof. Tomas Olofsson.

\section*{Kontor 1}

Utredarna ser ett mörkt rum, fyllt med kartonger och diverse bråte. De mörkt röda persiennerna är nerdragna, och när ni känner på dörren är den låst. Om Utredarna väljer att dyrka upp dörren är det ett svårt slag, då universitetet använder säkra lås för sina kontor. I kontoret kan dock utredarna finna dokument och mindre verktyg som korresponderar med professorns fält, samt en en mindre kniv använd för att öppna kartonger samt brev.

\section*{Kontor 2}

Utredarna kommer till ett kontor med ett skrivbord centrerat i rummet. Bakom skrivbordet sitter professorn och läser i en tidskrift. Om någon undersöker tidskriften kan man lätt se att det är den senaste volymen av \textit{Luleå studentkårs magasin}. På skrivbordet står en skrivmaskin och en massa papper samt böcker är utspridda runt omkring.

\section*{Kontor 3}
Utredarna ser ett rum med skrivbord, hylla samt en soffa som alla är intryckta emot väggarna. I mitten av rummet står någonting relaterat till professorns fält. Det kan vara:
%
\begin{itemize}
	\item En stor maskin som är avslagen.
	\item Ett terrarium för djur/växter
	\item En ofärdig staty
	\item En elektrisk maskin, med massa reläer och lampor (En gammaldags dator)
\end{itemize}

I rummet håller professorn på med sitt projekt.


\section{Prof. Migele Chapdelains kontor}
\label{loc:ChapdelainsKontor}
%
Professor Chapdelains är för närvarande inte på sitt kontor, men utredarna kan lätt upptäcka att dörren är olåst. Kontoret verkar vara mycket spartanskt. Det finns en tom bokhylla, ett skrivbord, en kontorsstol samt ett konstverk.

\begin{displayquote}
	Tavlan föreställer ett lila moln, eller kanske en vortex, som verkar strömma ut ur målningen. I botten på tavlan kan man se siluetter av människor som verkar utföra någon form av ceremoni eller tillbedjan. Verket är omslutet av en mörk träram.
\end{displayquote}

\begin{displayquote}
	Skrivbordet är en massiv pjäs i mörkt trä. Det står ut mot rummet, och är tomt. Både kontorsstolen samt bokhyllan är gjorda i samma stil som bordet, och är lika tomma även dem.
\end{displayquote}
%
Med ett normalt lyckat slag emot intelligens inser utredarna att både skrivbordet och bokhyllan är täckta i ett tjockt lager damm. Om utredarna kollar i en av de två skrivbordslådorna hittar de där en gammal bibel. Om utredarna vill undersöka bokhyllan så är den för hög för att man ska kunna se ovansidan, så de behöver antingen assistans från en annan utredare, eller något att stå på för att göra en genomgående undersökning. Lyckas denne slå ett lyckat svårt slag för finna dolda ting när de kan se hela bokhyllan hittar de en bit tejp på den översta hyllan som verkar fästa i någonting på baksidan av bokhyllan. Med ett lyckat normalt slag för fingerfärdighet kan de fiska upp den lapp som sitter på baksidan av bokhyllan. De kan även försöka flytta bokhyllan med ett normalt lyckat slag, och på så sätt få tag i lappen. Lappen är densamme som Migele har i rockfickan när de träffar på honom på Brådmans Bar \sectiondescribe{\ref{loc:BradmansBar}}.
