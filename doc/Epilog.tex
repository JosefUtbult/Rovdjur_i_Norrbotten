\section{Epilog}
Lyckas Utredarna att föra leopardkniven till polishuset, och visa \textit{Carl Svidsstål} 
\sectiondescribe{\ref{kar:KonstapelCarlSvidstal}} denna, har polisen nog grund för att anhålla 
Professor Olofsson. De tar sig till professorn hus för att arrestera professorn. Olofsson gör 
inget motstånd och följer med till polishuset där han sätts i en cell. Månader efter 
uppmärksammar ni en artikel i Luleå lokaltidning.

\begin{displayquote}
	Professor Thomas Olofsson hittades tidigt i morse liggande i parken utanför Luleå Universitet,
	rabblande och till synes spritt språngande galen efter den natt av oklara oväder runt området.
	Professorn, som tidigare i veckan blivit villkorligt frigiven för att vänta rättegång i fallet 
	angående hans systerdotter Anna, har nu lagts in på Sunderby institut. I hans tillstånd kommer
	professorn inte längre kunna delta i den rättegång som hålls emot honom, utan han kommer
	representeras fullt ut av sin advokat.

	Orsaken till professorns plötsliga infall är ännu oklart, men många individer som varit i 
	området under natten kan vittna om ``ovanliga ljusfenomen'' och ``odjur som kryper i 
	skog och snår''. Polismyndigheterna undersöker för tillfället dessa vittnen och har get 
	kommentar. 

	``Vi vet inte exakt vad som har hänt, men vi kommer att gå ut med vårt resultat så fort de 
	kan klargöras'' - Kommissarie Svidstål.
\end{displayquote}

Det som har hänt är att professorn har undergått riten beskriven under \textit{Åkallelsen} 
\sectiondescribe{\ref{sek:Akallelsen}}, vilket resulterat i hans oåterkalleliga galenskap.
Det spelarna dock inte vet är att denna ritual inte bara påverkat professorn, utan också 
de varelser som lever i våra svenska skogar.
